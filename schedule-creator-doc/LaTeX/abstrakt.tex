\documentclass[a4paper]{article}
\usepackage[utf8]{inputenc}                                      
\usepackage[OT4]{fontenc}  
\usepackage[polish]{babel} 
\usepackage{indentfirst}
%\usepackage{graphicx} 
%\usepackage{url}

\usepackage[margin=1.5in]{geometry}
\linespread{1.3}

\frenchspacing
%\let\cleardoublepage\clearpage  % po co?



\title{\textbf{[Tytuł pracy]}\\ \large abstrakt pracy dyplomowej inżynierskiej}
\author{Imię Nazwisko}


\begin{document}

\maketitle
\thispagestyle{empty}

Aplikacja przeznaczona jest pod komputery osobiste z systemem Windows 7 (lub nowszy) i została zaprojektowana z myślą o osobach korzystających z czytników e-booków. Jej celem jest ułatwienie zarządzania plikami zakupionych wcześniej książek elektronicznych z poziomu PC. Projekt został stworzony przy użyciu platformy .NET w języku C\#. W trakcie implementacji użyto silnika graficznego WPF oraz wzorca architektonicznego MVVM. Program został zaprojektowany jako system wieloużytkownikowy i wymaga założenia konta. Aplikacja pozwala tworzyć wirtualne biblioteki, do których mogą być dodawane książki elektroniczne w formatach EPUB, MOBI oraz PDF -- z nich odczytywane są dane, które później przechowuje się w bazie danych (utworzonej w systemie SQL Server). Użytkownik posiada dostęp do własnych kolekcji z dowolnej maszyny, na której znajduje się aplikacja dzięki możliwości synchronizacji (pobrania plików z bazy na dysk komputera). Program posiada opcję edycji metadanych książek znajdujących się w bibliotece oraz funkcjonalności dodatkowe (np. bezprzewodowy transfer e-booków na czytnik lub konwersję z formatu EPUB na MOBI).


\end{document}